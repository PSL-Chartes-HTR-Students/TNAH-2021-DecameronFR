\documentclass{article}
\usepackage{geometry}
\geometry{
	a4paper,
	total={170mm,257mm},
	left=20mm,
	top=20mm,
}
\usepackage[utf8]{inputenc}
\usepackage[italian, english, french]{babel}
\usepackage[TS1,T1]{fontenc}
\usepackage{lmodern}
\DeclareUnicodeCharacter{17F}{{%
		\fontencoding{TS1}%
		\selectfont s%
}}
\newcommand\textvtt[1]{{\normalfont\fontfamily{cmvtt}\selectfont #1}}

\usepackage{hyperref}
\usepackage[acronym,toc]{glossaries}
\makeglossaries
\newacronym{cremma}{CREMMA}{Consortium Reconnaissance d’Écriture Manuscrite des Matériaux Anciens}
\newacronym{htr}{HTR}{\textit{Handwritten Text Recognition}}
\newacronym{mufi}{MUFI}{\textit{\textit{The Medieval Unicode Font Initiative}}}

\usepackage[sorting=nyt,style=enc]{biblatex}
\usepackage{csquotes}
\usepackage{adjustbox}
\usepackage{tikz}
\usetikzlibrary{arrows.meta,
	chains,
	positioning,
	shapes.geometric}

\addbibresource{decameron.bib}

\newcommand{\siecle}[1]{\textsc{#1}\ieme}

\title{decameron-rapport}
\author{Sébastien Biay, Jean-Victor Boby, Zoé Cappe, Kristina Konstantinova}
\date{20 janvier 2022}

\begin{document}
	
	\maketitle
	
	\section{Présentation du projet}
	\subsection{Un projet de sources numériques sur Github/eScriptorium}
	Le projet \textbf{Decameron-FR} est un travail collectif dédié à l'\gls{htr}, qui se base sur l’utilisation du Git et Github pour la mise en commun du travail produit par chacun des membres du groupe, et de eScriptorium pour l’entraînement de modèles de reconnaissance automatique des écritures manuscrites. 
	
	L’intérêt de l’utilisation de Git dans ce projet de groupe était de pouvoir travailler à distance sur un dépôt commun et conserver un historique des modifications qui y ont été apportées, et Github nous a permis de communiquer sur les différents problèmes à résoudre, et vérifier les apports de chacun afin de pouvoir mieux les partager. Le choix de la plateforme eScriptorium a été fait car elle a l’avantage d’être un projet open source, en ligne, et permet de travailler aisément à plusieurs sur un même projet.
	
	Afin d’entraîner un modèle de modèles de reconnaissance automatique des écritures manuscrites, nous avons choisi de travailler sur un manuscrit du XVe siècle, qui présente des défis de lecture. Un travail préalable a donc été nécessaire, afin d’établir des normes concernant la transcription des caractères particuliers et de la ponctuation, et une procédure à suivre pour la segmentation du texte. Cela a imposé un choix parmi des modèles à entraîner, qui  s’est finalement porté sur deux d’entre eux: le modèle CREMMA-Médiéval 1.0.0 et le modèle fineTunEneide.
	
	\subsection{Choix de la source et des modèles}
	
	\subsubsection{Le manuscrit Paris, Arsenal, réserve 5070}
	Plusieurs membres du groupe ayant eu l'occasion de travailler sur un texte commun en première année de master TNAH – la traduction française du \textit{Decameron} de Boccace par Laurent de Premierfait –, nous avons choisi de pousser ce sujet plus loin.
	Le manuscrit que nous avions transcrit étant conservé à la Biblioteca Apostolica Vaticana (Pal. lat. 1989), dont les images diffusées en ligne ne sont pas libres de droits, nous nous sommes reportés sur le manuscrit Paris, Bibliothèque de l'Arsenal, Ms-5070 réserve, accessible sur Gallica\footcite{gallicaParisBibliothequeArsenal, bnfarchivesetmanuscritsMs5070BoccaceDecameron} et dont le colophon précise qu'il a été copié sur le fameux manuscrit du Vatican\footcite{jonas-irhtParisBibliothequeArsenal2012}.
	
	Le ms. 5070 de l'Arsenal date du deuxième quart du \siecle{xv} siècle. Son écriture se rattache à la catégorie de la \textit{littera cursiva} selon la classification de Lieftinck-Gumbert-Derolez, catégorie décrite ainsi par D. Stutzmann~:
	\begin{quote}La \textit{littera cursiva} se caractérise par un \textit{a} à simple ove, des \textit{f} et \textit{s} longs filant sur la ligne, et des lettres à hastes avec des boucles. Elle est donc à l’opposé de la \textit{littera textualis}\footcite{stutzmannEcrituresGothiquesLivresques2022}.\end{quote}
	
	Son écriture, très régulière, est le fait d'une seule main. Le colophon révèle l'identité du copiste~: \begin{quote}
		\textit{Explicit la table du transcripvain Guillebert de Mets, hoste de l'Escu de France a Gramont}\footcite{jonas-irhtParisBibliothequeArsenal2012}.
	\end{quote}Né vers 1390 à Grammont (Flandres), Guillebert de Mets exerça l'activité de copiste pour les ducs de Bourgogne à Paris\footcite{jonas-irhtGuillebertMets}. 
	
	\subsubsection{Modèles HTR pour les manuscrits littéraires médiévaux et spécificités de leurs corpus d'entraînement}
	\paragraph{Cremma Médiéval 1.0.0 Bicerin}
	Le premier modèle \gls{htr} que nous avons choisi est l'un des deux modèles entraînés dans le cadre du projet \gls{cremma} sur les manuscrits médiévaux du \siecle{xii} au \siecle{xiv} siècle : 1.0.0 Bicerin\footcite{pincheCREMMAMedievalOld2021}.
	
	\paragraph{Modele fineTunEneide}
	Le second modèle que nous avons entraîné est une personnalisation du modèle Bicerin 1.0.0 à partir de l'écriture du manuscrit Philadelphie, University of Pennsylvania, Rare Book and Manuscript Library, Codex 909, réalisée par Lucien Dugaz dans le cadre d'une recherche post-doctorale au sein du Centre Jean Mabillon (EA 3624). Écrit en France dans la seconde moitié du \siecle{xv} siècle, ce manuscrit est plus proche d'Arsenal 5070 par sa chronologie, et son écriture est de type gothique bâtarde\footcite{dugazEditionCritiqueNumerique2021, pennlibrariesMedievalRenaissanceManuscripts}.
	
	\subsubsection{Ontologie de segmentation des zones~: Segmonto}
	Pour réaliser la segmentation en zones des pages de notre manuscrit, nous avons suivi les propositions de typage des zones formulées dans les \textit{Guidelines} du projet SegmOnto, dont le vocabulaire contrôlé couvrait largement nos besoins\footcite{campsSegmOntoGuidelines2021}.
	
	\section{Mise en oeuvre}
	Les sources choisies et le sujet défini, la planification du projet a constitué notre première étape, afin d’élaborer le plan d'exécution. Cette phase essentielle a permis d'identifier les tâches successives à réaliser, leur ordre, leur répartition et leurs délais respectifs. Elle sert également à s’accorder sur le fonctionnement de l’équipe, les responsabilités de chacun et les méthodes de conduite et de suivi du projet.
	\subsection{Organisation fonctionnelle et aspects techniques}
	
	\subsubsection{Git et GitHub:}
	
	Un dépôt distant GitHub \footnote{Voir TNAH-2021-DecameronFR à \href{https://github.com/PSL-Chartes-HTR-Students}{https://github.com/PSL-Chartes-HTR-Students}.} est créé et partagé pour la gestion du projet. Les membres de l'équipe traitent chacune des tâches qui leur sont assignées au sein d'une branche dédiée dans leurs dépôts locaux, clonés avec Git\footnote{Git version 2.25.1.} à l'aide des commandes du terminal comme indiqué dans le schéma ci-dessous :
	\vspace{3pt}
	\begin{figure}[h!]
		\centering
		\begin{tikzpicture}[
			node distance = 7mm,
			box/.style = {draw=black, fill=red!30, align=center, minimum height=7mm},
			boxr/.style = {ellipse, draw=black, fill=orange!30, minimum width=6mm, minimum height=9mm, inner xsep=0pt, inner ysep=1pt, align=center},
			inp/.style = {trapezium, trapezium left angle=70, trapezium right angle=110, minimum width=6mm, minimum height=7mm, draw=black, fill=blue!30},
			inp2/.style = {trapezium, trapezium left angle=70, trapezium right angle=110,  text centered, draw=black, fill=blue!30, anchor=north, text depth = 13mm}, trapezium stretches=true,
			fichier/.style = {diamond, text centered, draw=black, fill=green!30}
			]
			\node[font=\large] (n1) [box]  {Commandes\\principales GitHub};
			\node[font=\large] (n2) [boxr,right of=n1, xshift=3cm]   {Dépôt\\ distant};
			\node[font=\large] (n3) [boxr,right of=n2, xshift=3cm]   {Dépôts\\ locaux};
			\node[font=\large] (n4) [box,right of=n3,xshift=3cm]   {Commandes\\ principales git};
			\node[font=\large] (n5) [text width=5cm, align=center, below of=n4, yshift=-0.5cm] {git pull};
			\node[font=\large] (n6) [text width=6cm, align=center, below of=n5, yshift=-0.8cm] {git checkout -b Readme\\Création de readme.md\\git add\\git commit -m "message"};
			\node[font=\large] (n7) [text width=2.5cm, align=center, below of=n1, yshift=-4.2cm] {pull request \\ merge};
			\node (n8) [inp, below of=n2, yshift=-0.5cm] {Main};
			\node (n9) [inp2, minimum height=1.5cm, below of=n3, yshift=-2cm] {Readme};
			\node (10) [fichier, below of=n3, yshift=-2.2cm] {.md};
			\node[font=\large] (n11) [text width=5cm, align=center, below of=n4, yshift=-3.3cm] {git push (-u origin master)};
			\node (n12) [inp, minimum width=28mm, below of=n2, yshift=-3.3cm] {Readme};
			\node (n13) [inp, below of=n2, yshift=-2.5cm] {Main};
			\node (n14) [diamond, draw=black, fill=green!30, below of=n2, xshift=1cm, yshift=-3.3cm]{};
			\node (n15) [inp, minimum width=22mm, below of=n2, yshift=-4.2cm] {Main};
			\node (n16) [diamond, draw=black, fill=green!30, below of=n2, xshift=0.7cm, yshift=-4.2cm]{};
			%
			\draw[>=latex, <->]  (n2) -- (n3);
			\draw[>=latex, ->]  (n8) -- (n5);
			\draw[>=latex, ->]  (n11) -- (n12);
			\draw[>=latex, ->]  (n7) -- (n15);
		\end{tikzpicture}
		\caption{Processus de création d'un readme.md pour le dépôt} \label{fig1}
	\end{figure}
	\vspace{3pt}
	
	Lorsqu'une tâche est achevée, les mises à jour sont envoyées sur le dépôt distant afin que l'équipe les valide et fusionne les modifications vers la branche principale (\textit{main}), le cas échéant. Chaque thème important fait l'objet de développements dans une branche distincte\footnote{Voir la liste des branches sur le dépôt.}, notamment pour l'établissement des normes et les transcriptions individuelles.
	\newline
	
	
	La fonctionnalité des \textit{issues} de GitHub est quant à elle dédiée aux échanges de l'équipe, également par sujets. Autant que possible, un lien est établi entre les branches, la documentation qu'elles contiennent et les issues.
	Ces dernières doivent être rédigées de manière claire et illustrées, en particulier sur les points de discussion relatifs aux transcriptions et à leurs normes.
	
	\subsubsection{eScriptorium}
	
	Un projet commun a été créé sur eScriptorium, sur lequel les images qui avaient été sélectionnées pour la transcription ont été importées. L’import par le manifeste IIIF a été préféré à un PDF ou des fichiers locaux, afin d’obtenir la meilleure qualité d’image possible.
	
	Les pages ont d’abord été segmentées automatiquement, puis cette segmentation a été corrigée manuellement, notamment pour la séparation des zones de texte et de lignes (colonnes, titre courant, notes en marge, lettrines), la suppression des zones et lignes inutiles (texte des pages suivantes apparaissant sur le côté des images) et pour la numérotation des lignes (colonnes et titre courant). Les zones ont pu être nommées selon les propositions de SegmOnto, après leur ajout dans l’onglet «Description».
	
	Une fois la segmentation terminée, les deux modèles (Bicerin 1.0.0 et fineTunEneide) ont pu être entraînés. La transcription a ensuite été faite dans l’onglet \textit{manual} des modèles, et la transcription multi-ligne a été utilisée~: en plus d’avoir l’avantage de mettre en surbrillance la ligne qui est en cours de transcription, elle permet d’utiliser aisément le clavier virtuel qui a été importé depuis le dépôt commun. Les transcriptions ont ensuite été exportées au format XML (selon le standard ALTO) depuis eScriptorium, et déposées par chacun dans le dépôt grâce à Git.
	
	Les deux modèles ont alors pu être entraînés à partir des transcriptions, et les modèles réentraînés ont pu être téléchargés depuis l’interface et ajoutés au dépôt de la même façon que les transcriptions.
	
	\subsection{Préparation et réalisation de la transcription}
	\subsubsection{Etablissement des normes de transcription}
	Les normes de transcription que nous avons adoptées et qui sont détaillées dans le fichier \textvtt{normesTranscription.md} s'efforcent de suivre celles définies dans le cadre du projet \gls{cremma}-Médiéval, elles mêmes inspirées de la \textit{graphemic transcription} telle que conçue par D. Stutzmann\footnote{Voir le fichier \texttt{Lisez-moi} du dépôt d'\cite{pincheCREMMAMedievalOld2021}~; \cite{stutzmannPaleographieStatistiquePour2011}}. Les fautes n'ont pas été corrigées ni les abréviations résolues. Nous avons occasionnellement fait appel aux caractères proposés par \gls{mufi}\footcite{MedievalUnicodeFont2016}. Les choix d'encodage correspondant à chaque abréviation sont listés dans le fichier \texttt{caracteres.html}.
	
	\subsubsection{Le problème des i/j et u/v}
	L'application des principes de la transcription graphématique a rencontré quelque diffilcuté à propos des caractères i/j et u/v. 
	\paragraph{Hétérogénéité du corpus \gls{cremma}-Médiéval}
	Les vérités de terrain du projet \gls{cremma}-Médiéval présentent une inégalité de traitement : i/j et u/v y sont parfois distingués selon leur valeur phonétique malgré l'absence de distinction graphique ; ils sont parfois distingués selon leur forme graphique indépendamment de leur valeur phonétique\footnote{Voir les exemples du document \texttt{normesTranscription.md}}.
	
	\paragraph{La transcription graphématique~: une définition difficile à saisir}
	La définition de la \textit{graphemic transcription} proposée dans l'article de D.~Stutzman cité en référence par le projet \gls{cremma}-Médiéval\footcite[p.~251]{stutzmannPaleographieStatistiquePour2011} ne nous a pas permis de lever cette difficulté de manière définitive. L'auteur écrit que la tradition philologique aurait déjà clarifié les problèmes de la \textit{graphemic transcription}, citant les \textit{Conseils pour l'édition des textes médiévaux} publiés par l'École des chartes\footcite{ecolenationaledeschartesConseilsPourEdition2001a}. Pourtant, cette publication prône essentiellement des normes de transcription de type interprétative pour les textes littéraires~; les abréviations y sont restituées et les lettres i/j et u/v distinguées, en l'absence de doute selon leur valeur phonétique et non graphique\footcite[fasc. 1, p.~23 et suivantes, ainsi que 31 pour les abréviations]{ecolenationaledeschartesConseilsPourEdition2001a}.
	
	\paragraph{Choix~: avantages et inconvénients}
	Nous avons fait le choix de ne distinguer i/j et u/v ni selon leur valeur, ni selon leur forme. La limite de ce choix de ne pas distinguer i/j et u/v est certes de rendre impropre nos données à l'étude de l'apparition des lettres v et j dans les manuscrits médiévaux. Ce choix a en revanche pour avantage sa simplicité de conception, et reste en cohérence avec le système graphique du manuscrit dans la mesure où il ne l'affecte que sur les caractères initiaux : le scribe a employé les caractères j et v exclusivement et systématiquement en position initiale ; nous les avons toujours transcrits i et u. Ce choix n'interprétant pas la valeur phonétique des caractères, il ménage ainsi la possibilité d'une évolution des normes.
	
	\subsubsection{Utilisation de la plateforme eScriptorium}
	La mise en page très canonique (texte sur deux colonnes) et l'absence d'ajout au texte sur les pages que nous avons traitées ont rendu aisée l'étape de la segmentation des folios, bien que la prise en main de la plateforme eScriptorium ne se soit avérée quelque peu déroutante.
	
	\paragraph{Segmentation}
	La fonction dessiner une ligne étant par défaut activée, le moindre clic gauche pour sélectionner un objet s'est avéré périlleux ! La segmentation automatique peine a distinguer clairement les deux colonnes de texte, à redécouper dans la plupart des cas. Autre problème, heureusement peu fréquent, les bouts de ligne de la page en regard, qui sont susceptibles d'apparaître en bordure de la photographie d'une page, ne sont pas d'une suppression évidente, et tenter de sélectionner ces lignes pour les supprimer conduit souvent à créer de nouvelles lignes, jusqu'à ce que l'on affiche la zone concernée à 800\% de sa taille réelle pour sélectionner les embryons de lignes indésirables.
	
	\paragraph{Transcription}
	L'entraînement du modèle \gls{htr} a permis de constater de très nombreuses confusions entre ſ et l, dont résulte la transcription de nombreux l en s. Ces confusions ont confirmé l'importance de la distinction des s longs et des s ronds que nous avions retenue parmi les normes de transcription du projet, par-delà l'argument de la stabilité des choix scribaux sur la longue durée, qui fonde a priori la distinction des allographes de s.
	
	Notre texte étant particulièrement proche de celui de la Bibliothèque vaticane (composé dans une écriture plus livresque), la confrontation des manuscrits nous a permis de lever toutes les difficultés de lecture.
	
	L'utilisation du clavier virtuel s'est avérée salutaire pour la saisie des caractères spéciaux définis dans notre table des caractères. En revanche, nous avons souvent été confronté à un bogue consistant en l'insertion du caractère sélectionné sur le clavier virtuel non pas à l'emplacement de la saisie en cours mais au début de la ligne. Ce problème survient lors de la saisie dans une ligne donnée d'un premier caractère au clavier virtuel, et non pour les caractères suivants. Ce premier caractère effacé, les suivants seront insérés comme il se doit à l'emplacement du curseur. Mais le problème se représentera dès la ligne suivante !
	
	Autre facétie de la plateforme~: la numérotation des lignes transcrites revenant entre deux sessions de travail à sa disposition initiale en dépit de sa modification manuelle. En effet, les titres courants ont parfois dû être placés plusieurs fois en position intiale, la machine s'obstinant à les placer après la numérotation des lignes de la première colonne.
	
	\section{Bilan}
	
	\subsection{Réflexion sur l'organisation fonctionnelle et les aspects techniques}
	\subsubsection{Git et GitHub}
	D'un point de vue technique, après une période d'adaptation nécessaire à toute nouvelle pratique, les outils Git et GitHub se prêtent bien à la collaboration simultanée sur le projet. L'équipe semble maîtriser les processus de mise à jour et/ou de correction lorsque cela s'est avéré nécessaire (le facteur humain induit inévitablement des erreurs ponctuelles). Sans doute, et surtout pour des projets plus conséquents, aurait-il fallu créer une branche "Dev" de développement "tampon" entre les branches et le "main", pour imiter les risques de corrompre ce dernier.
	
	
	Concernant la conduite du projet, force est de constater qu'il reste essentiel d'organiser des réunions de projet pour discuter de certains points, comme ceux qui sont trop complexes à expliciter exclusivement par écrit ou ceux qui n'ont pas vocation à être publiés. A elles seules, les fonctionnalités de discussion et de suivi permises via les "issues" ou les discussions GitHub ne sont pas suffisantes.
	A propos des "issues", il aurait été préférable d'ouvrir une nouvelle "issue" sur chaque question dès le départ, notamment s'agissant des caractères : l'"issue" n° 11 a rassemblé les questionnements concernant trop de caractères différents. Par la suite, il est devenu difficile d'y retrouver une information. Enfin, GitHub ne semble pas permettre de déplacer un commentaire, d'une "issue" vers une autre (voire vers une nouvelle) qui lui correspondrait mieux, contrairement à de nombreux outils en ligne. Il semble seulement possible de faire des références.
	
	
	\subsubsection{Evaluation de la plateforme eScriptorium}
	
	La plateforme eScriptorium est un outil qui permet de travailler sans difficulté en groupe, y compris simultanément; cela demande néanmoins une certaine organisation entre les membres du groupe, afin de ne pas défaire ce qui a pu déjà être fait par d’autres, en procédant à la segmentation d’une partie déjà faite (et ainsi en écrasant toutes les corrections déjà faites).
	
	L’apprentissage de l’utilisation de la plateforme peut être assez complexe au début, mais une fois cette étape passée, elle est facile à manipuler, que ce soit pour la segmentation ou la transcription. Il peut cependant y avoir des petits problèmes qui demandent des corrections supplémentaires, même après avoir finalisé la segmentation et commencé la transcription, tels que les problèmes de positionnement du curseur lors de l’insertion des caractères du clavier virtuel ou de numérotation du titre courant mentionnés ci-dessus (il a fallu une fois se résoudre à segmenter à nouveau une page afin de pouvoir replacer correctement la ligne), voire de décalage de lettres dans la transcription qui nécessitent une relecture complète de la page. 
	
	Ces problèmes ont été résolus sans trop de difficultés sur notre petit nombre de pages (neuf en tout), mais sont beaucoup plus importants si l’on envisage de travailler sur un corpus beaucoup plus volumineux.
	
	Il est aussi très intéressant de pouvoir choisir le modèle le plus adapté à l’écriture de son texte, tous les modèles déjà entraînés étant disponibles sur la plateforme, ainsi que de pouvoir entraîner son propre modèle, à partir d’un modèle déjà existant ou non.
	
	\textbf{Autres idées de Victor: lors de la segmentation, les ombres sur le pli intérieur du feuillet est parfois interprété comme une Zone (même problèmes avec Tesseract pour Python) qui doit être supprimée. Cette suppression ne fonctionne pas toujours convenablement et des Zones de l'ordre de quelques pixels peuvent rester. Il faut bien penser à exclure les Zones non définies et les lignes orphelines lors de l'export Alto. La sélection des zones/lignes est très difficile et nécessite beaucoup de précision.}
	
	
	\subsubsection{Retour d'expérience}
	\subsection{Volet sources:}
	\subsubsection{Performances des modèles entraînés}
	\subsubsection{Réflexion sur les normes de transcription}
	\textbf{Dire simplement que des recommandations générales pour les projets de ce type seraient les bienvenues.}
	
	\printglossaries
	
	\printbibheading
	
	\printbibliography[heading=subbibliography,title=Projets et ressources,keyword=projets]
	
	\printbibliography[heading=subbibliography,title=Notices,keyword=notices]
	
	\printbibliography[heading=subbibliography,title=Études,keyword=autres]
	
\end{document}
